\documentclass[]{beamer}
    \usepackage[utf8]{inputenc}
    \usepackage[czech]{babel}
    %\usepackage{times}
    %\usepackage[hidelinks]{hyperref}
    \usepackage{graphicx}
        \graphicspath{{./images/}}
    \usepackage{float}
    \usepackage[dvipsnames]{xcolor}
    \usepackage{tikz}

    \usepackage[outputdir=build]{minted}
    \usemintedstyle{friendly}

    \usetheme{Madrid}
    \usecolortheme{default}
    %\usefonttheme{wolverine}
    \setbeamertemplate{blocks}[rounded][shadow=true]

    \title[Základy obfuskace]{Základy obfuskace}
    \subtitle[]{aneb Jaký užitek má nečitelný kód?}
    \author[]{Bc.\ Martin Jabůrek}
    \institute[PRG PRG]{Student Programming Language Meetup}
    \date[]{28.\ listopadu\ 2025}

\newcommand{\image}[3]{ % args: size, path, caption
    \begin{figure}[H]
        \centering
        \includegraphics[width=#1\textwidth]{#2}%\caption{#3}
        \label{fig:\thefigure}
    \end{figure}
}

\newcommand{\Image}[5]{ % args x, y, scale, path, caption
    \begin{tikzpicture}[remember picture, overlay]
        \node at (#1, #2) { \includegraphics[scale=#3]{#4} };
        % \node[below] at (#1, #2) {#5};
    \end{tikzpicture}
}

\renewcommand{\t}[1]{\texttt{#1}}

\begin{document}

\frame{\titlepage}

\begin{frame}
    \frametitle{Moje bakalářka}

    \Image{6}{0}{0.25}{screen_VUT_web.png}
    
\end{frame}

\begin{frame}
    \frametitle{Jak to dopadlo?}

    \begin{itemize}
        \pause{}
        \item Překladač \dots
        \pause{}
        \item \dots generující cílový kód \dots
        \pause{}
        \item \dots odolný proti zpětné analýze
    \end{itemize}

\end{frame}

\begin{frame}
    \frametitle{Zpětná analýza}

    \begin{itemize}
        \item Proč zpětná analýza?
        \pause{}
        \item Jaké máme pro zpětnou analýzu prostředky?
        \pause{}
        \begin{itemize}
            \item Disasseblery
            \pause{}
            \item Debuggery
            \pause{}
            \item Dekompilátory
            \pause{}
            \item Tracing nástroje
            \pause{}
            \item Pathing nástroje
            \pause{}
            \item Dumping nástroje
        \end{itemize}
    \end{itemize}

\end{frame}

\begin{frame}
    \frametitle{Ochrana proti zpětné analýze}

    \begin{itemize}
        \item Proč se proti ní potřebujeme bránit?
        \pause{}
        \item Co je cílem ochrany proti zpětné analýze?
        \pause{}
        \item Jaké máme metody ochrany proti zpětné analýze?
        \pause{}
        \begin{itemize}
            \item Odstraňování symbolických informací
            \pause{}
            \item Šifrování kódu
            \pause{}
            \item Antidebugging
            \pause{}
            \item SaaS
            \pause{}
            \item \textbf{Obfuskace}
        \end{itemize}
    \end{itemize}

\end{frame}

\begin{frame}
    \frametitle{Obfuskace}

    \textbf{Obfuskace} kódu představuje jednosměrnou transformaci kódu za zachování
    jeho pozorovatelného chování. Jejím primárním účelem je učinit kód složitějším,
    obtížněji čitelným a pochopitelným.

\end{frame}

\begin{frame}
    \frametitle{Způsoby obfuskace}

    Obfuskace:
    \begin{itemize}
        \pause{}
        \item zdrojových kódů
        \pause{}
        \item bytekódů
        \pause{}
        \item binárních kódů
    \end{itemize}

\end{frame}

\begin{frame}
    \frametitle{Hodnocení obfuskace}

    \begin{itemize}
        \item Síla obfuskace\,--\,\emph{potency}
        \pause{}
        \item Skrytost\,--\,\emph{concealment}
        \pause{}
        \item Odolnost\,--\,\emph{resilience}
        \begin{itemize}
            \pause{}
            \item Proti manuální zpětné analýze
            \pause{}
            \item Proti automatizované zpětné analýze
        \end{itemize}
        \pause{}
        \item[ ]
        \item Cena
    \end{itemize}

\end{frame}

\begin{frame}
    \frametitle{Různé druhy obfuskace}

    \begin{itemize}
        \pause{}
        \item Transformace toku řízení
        \begin{itemize}
            \pause{}
            \item Výpočetní transformace\,--\,\emph{Computation transformation}
            \pause{}
            \item Seskupovací transformace\,--\,\emph{Aggregation transformation}
            \pause{}
            \item Transformace pořadí\,--\,\emph{Ordering transformation}
        \end{itemize}
        \pause{}
        \item Transformace dat
        \pause{}
        \item Další netradiční techniky\dots{}
    \end{itemize}

\end{frame}



\begin{frame}
    \frametitle{Neprůhledné predikáty}

    \emph{Opaque predicates}
\end{frame}

\begin{frame}
    \frametitle{Vkládání mrtvého kódu}
\end{frame}

\begin{frame}
    \frametitle{Klonování kódu}
\end{frame}

\begin{frame}
    \frametitle{Zploštění toku řízení}

    Základní bloky

    \emph{control flow flattening}\\
    \emph{table interpretation}
\end{frame}

\begin{frame}
    \frametitle{Úprava cyklů}

    \begin{itemize}
        \item Loop blocking
        \item Loop unrolling
        \item Loop fission
    \end{itemize}
\end{frame}

\begin{frame}
    \frametitle{Prokládání kódu}
\end{frame}

\begin{frame}
    \frametitle{Umělé rozšíření výpočtu}

    \begin{itemize}
        \item Změna kódování dat
        \item Změna struktury dat
        \item Expanze literálů
        \item Rozšíření podmínek
        \item Redundantní parametry funkcí
    \end{itemize}
\end{frame}

\begin{frame}
    \frametitle{Obfuskace optimalizacemi}

    \begin{itemize}
        \item Střídání instrukcí
        \item Inlining
        \item Optimální algoritmy
        \item Paralelizace kódu
    \end{itemize}

    Opak inliningu \dots{} Outlining
\end{frame}

\begin{frame}
    \frametitle{Úprava symbolických informací}

    \begin{itemize}
        \item Odstranění knihovních volání
        \item Šifrování
        \item Kódovaná pojmenování
        \item Splývající pojmenování
        \item \textbf{Falšování symbolických informací}
    \end{itemize}
\end{frame}

\begin{frame}
    \frametitle{\textbf{Obfuskace znaménkovsti a datového typu}}

\end{frame}

\begin{frame}
    \frametitle{\textbf{Jazyk samotný}}

    \begin{itemize}
        \item \t{foreach}
        \item \t{redo}
        \item \t{restart}
    \end{itemize}

    Reducibilní / Ireducibilní CFG
\end{frame}



\begin{frame}[fragile]{Side-by-Side Assembly Examples}

    \begin{columns}[T,onlytextwidth]

        \begin{column}{0.48\textwidth}
        \begin{block}{Example 1: Move and Add}
        \begin{minted}[fontsize=\scriptsize]{asm}
        .intel_syntax noprefix
        mov rax, 5
        add rax, 10
        mov rbx, rax
        \end{minted}
        \end{block}
        \end{column}

        \begin{column}{0.48\textwidth}
        \setbeamercolor{block title}{bg=green!60!black, fg=white}
        \setbeamercolor{block body}{bg=green!10, fg=black}
        \begin{block}{Example 2: Loop Structure}
        \begin{minted}[fontsize=\scriptsize]{asm}
        .intel_syntax noprefix
        mov rcx, 10
        loop_start:
            dec rcx
            jnz loop_start
        \end{minted}
        \end{block}
        \end{column}

    \end{columns}

\end{frame}

\begin{frame}
    \frametitle{Díky za pozornost!}

    \pause{}
    \centering{A jaké máte s obfuskací zkušenosti vy?}

\end{frame}

\end{document}
