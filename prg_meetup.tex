\documentclass[]{beamer}
    \usepackage[utf8]{inputenc}
    \usepackage[czech]{babel}
    %\usepackage{times}
    %\usepackage[hidelinks]{hyperref}
    \usepackage{graphicx}
        \graphicspath{{./images/}}
    \usepackage{float}
    \usepackage[dvipsnames]{xcolor}
    \usepackage{tikz}

    \usepackage[outputdir=build]{minted}
    \usemintedstyle{friendly}

    \usetheme{Madrid}
    \usecolortheme{default}
    %\usefonttheme{wolverine}
    \setbeamertemplate{blocks}[rounded][shadow=true]

    \title[Základy obfuskace]{Základy obfuskace}
    \subtitle[]{aneb Jaký užitek má nečitelný kód?}
    \author[]{Bc.\ Martin Jabůrek}
    \institute[PRG PRG]{Student Programming Language Meetup}
    \date[]{28.\ listopadu\ 2025}

\newcommand{\image}[3]{ % args: size, path, caption
    \begin{figure}[H]
        \centering
        \includegraphics[width=#1\textwidth]{#2}%\caption{#3}
        \label{fig:\thefigure}
    \end{figure}
}

\newcommand{\Image}[5]{ % args x, y, scale, path, caption
    \begin{tikzpicture}[remember picture, overlay]
        \node at (#1, #2) { \includegraphics[scale=#3]{#4} };
        % \node[below] at (#1, #2) {#5};
    \end{tikzpicture}
}

\renewcommand{\t}[1]{\texttt{#1}}

\begin{document}

\frame{\titlepage}

\begin{frame}
    \frametitle{Moje bakalářka}

    \Image{6}{0}{0.25}{screen_VUT_web.png}
    
\end{frame}

\begin{frame}
    \frametitle{Jak to dopadlo?}

    \begin{itemize}
        \pause{}
        \item Překladač \dots
        \pause{}
        \item \dots generující cílový kód \dots
        \pause{}
        \item \dots odolný proti zpětné analýze
    \end{itemize}

\end{frame}

\begin{frame}
    \frametitle{Zpětná analýza}

    \begin{itemize}
        \item Proč zpětná analýza?
        \pause{}
        \item Jaké máme pro zpětnou analýzu prostředky?
        \pause{}
        \begin{itemize}
            \item Disasseblery
            \pause{}
            \item Debuggery
            \pause{}
            \item Dekompilátory
            \pause{}
            \item Tracing nástroje
            \pause{}
            \item Pathing nástroje
            \pause{}
            \item Dumping nástroje
        \end{itemize}
    \end{itemize}

\end{frame}

\begin{frame}
    \frametitle{Ochrana proti zpětné analýze}

    \begin{itemize}
        \item Proč se proti ní potřebujeme bránit?
        \pause{}
        \item Co je cílem ochrany proti zpětné analýze?
        \pause{}
        \item Jaké máme metody ochrany proti zpětné analýze?
        \pause{}
        \begin{itemize}
            \item Odstraňování symbolických informací
            \pause{}
            \item Šifrování kódu
            \pause{}
            \item Antidebugging
            \pause{}
            \item SaaS
            \pause{}
            \item \textbf{Obfuskace}
        \end{itemize}
    \end{itemize}

\end{frame}

\begin{frame}
    \frametitle{Obfuskace}

    \textbf{Obfuskace} kódu představuje jednosměrnou transformaci kódu za zachování
    jeho pozorovatelného chování. Jejím primárním účelem je učinit kód složitějším,
    obtížněji čitelným a co nejhůře pochopitelným.

\end{frame}

\begin{frame}
    \frametitle{Způsoby obfuskace}

    Obfuskace:
    \begin{itemize}
        \pause{}
        \item zdrojových kódů
        \pause{}
        \item bytekódů
        \pause{}
        \item binárních kódů
    \end{itemize}

\end{frame}

\begin{frame}
    \frametitle{Hodnocení obfuskace}

    \begin{itemize}
        \item Síla obfuskace\,--\,\emph{potency}
        \pause{}
        \item Skrytost\,--\,\emph{concealment}
        \pause{}
        \item Odolnost\,--\,\emph{resilience}
        \begin{itemize}
            \pause{}
            \item Proti manuální zpětné analýze
            \pause{}
            \item Proti automatizované zpětné analýze
        \end{itemize}
        \pause{}
        \item[ ]
        \item Cena
    \end{itemize}

\end{frame}

\begin{frame}
    \frametitle{Různé druhy obfuskace}

    \begin{itemize}
        \pause{}
        \item Transformace toku řízení
        \begin{itemize}
            \pause{}
            \item Výpočetní transformace\,--\,\emph{Computation transformation}
            \pause{}
            \item Seskupovací transformace\,--\,\emph{Aggregation transformation}
            \pause{}
            \item Transformace pořadí\,--\,\emph{Ordering transformation}
        \end{itemize}
        \pause{}
        \item Transformace dat
        \pause{}
        \item Další netradiční techniky\dots{}
    \end{itemize}

\end{frame}



\begin{frame}[fragile]
    \frametitle{Neprůhledné predikáty}
    
    \vspace{4em}

    \begin{itemize}
        \item angl.\ \emph{Opaque predicates}
        \pause{}
        \item[ ]
        \item Falešné podmínky
        \pause{}
        \item Cykly o jedné iteraci
    \end{itemize}

    \pause{}

    \vspace{-11em}

    \begin{columns}[T,onlytextwidth]

        \begin{column}{0.48\textwidth}
        \end{column}

        \begin{column}{0.48\textwidth}
        \setbeamercolor{block title}{bg=green!60!black, fg=white}
        \setbeamercolor{block body}{bg=green!10, fg=black}
        \begin{block}{Příklad falešné podmínky}
        \begin{minted}[fontsize=\scriptsize]{asm}
        ...
        movq -8(%rbp), %rax
        negq %rax
        push %rax
        movq -8(%rbp), %rax
        notq %rax
        movq %rax, %rbx
        pop %rax
        subq %rbx, %rax
        test %rax, %rax
        cmovnz %r10, %rax
        test %rax, %rax
        jz __if_end_0002
        ...
        \end{minted}
        \end{block}
        \end{column}

    \end{columns}

\end{frame}

\begin{frame}
    \frametitle{Duplikace kódu}

    \begin{itemize}
        \pause{}
        \item Vkládání mrtvého kódu
        \begin{itemize}
            \pause{}
            \item Kód, který se nikdy neprovede
            \pause{}
            \item Efektivně mrtvý kód
            \pause{}
            \item Mrtvý kód v kontextu skutečného kódu
        \end{itemize}
        \pause{}
        \item Klonování kódu
    \end{itemize}

\end{frame}

\begin{frame}
    \frametitle{Úprava cyklů}

    \begin{itemize}
        \pause{}
        \item Rozdělování
        \pause{}
        \item Unrolling
    \end{itemize}
\end{frame}

\begin{frame}
    \frametitle{Zploštění toku řízení}

    \begin{itemize}
        \item angl.\ \emph{control flow flattening} nebo \emph{table interpretation}
        \item Úprava založená na tzv.\ základních blocích
        \item[ ]
        \item[ ]
        \item[ ]
        \item[ ]
        \item[ ]
        \item[ ]
        \item[ ]
    \end{itemize}

    \only<2>{
        \Image{6}{2}{0.75}{basic_blocks_1.drawio.pdf}
    }

    \only<3>{
        \Image{5.9}{2}{0.75}{basic_blocks_2.drawio.pdf}
    }

    \only<4>{
        \Image{5.9}{2}{0.75}{basic_blocks_3.drawio.pdf}
    }

    \only<5>{
        \Image{6}{2}{0.75}{basic_blocks_4.drawio.pdf}
    }

    \only<6>{
        \Image{6}{2}{0.75}{basic_blocks_5.drawio.pdf}
    }

\end{frame}

\begin{frame}
    \frametitle{Prokládání kódu}

    \pause{}

    \Image{6}{-0.5}{0.2}{spongebob.jpg}

\end{frame}

\begin{frame}
    \frametitle{Umělé rozšíření výpočtu}

    \begin{itemize}
        \pause{}
        \item Změna kódování dat
        \pause{}
        \item Změna struktury dat
        \pause{}
        \item Expanze literálů
        \pause{}
        \item Rozšíření podmínek
        \pause{}
        \item Redundantní parametry funkcí
    \end{itemize}
\end{frame}

\begin{frame}
    \frametitle{Obfuskace optimalizacemi}

    \begin{itemize}
        \pause{}
        \item Střídání instrukcí
        \pause{}
        \item Inlining\pause{} a outlining
        \pause{}
        \item Optimální algoritmy
        \pause{}
        \item Paralelizace kódu
    \end{itemize}

\end{frame}

\begin{frame}
    \frametitle{Úprava symbolických informací}

    \begin{itemize}
        \pause{}
        \item Odstranění knihovních volání
        \pause{}
        \item Šifrování
        \pause{}
        \item Kódovaná / plývající pojmenování
        \begin{itemize}
            \pause{}
            \item \_\_\_\_\_\_\_\_\_, \_\_\_\_\_\_\_\_, \_\_\_\_\_\_\_\_\_\_, \dots{}
            \pause{}
            \item llIIlIlI, lIlIIIll, IlIllIlI, \dots{}
        \end{itemize}
        \pause{}
        \item \textbf{Falšování symbolických informací}
    \end{itemize}

\end{frame}

\begin{frame}[fragile]
    \frametitle{\textbf{Obfuskace znaménkovsti a datového typu}}

    \vspace{-2em}

    \pause{}

    \begin{columns}[T,onlytextwidth]

        \begin{column}{0.48\textwidth}
        \begin{block}{\t{fibonacci.jk}}
        \begin{minted}[fontsize=\scriptsize]{asm}
            ...
        __for_start_0000:
            movq -32(%rbp), %rax
            push %rax
            movq COUNT(%rip), %rax
            movq %rax, %rbx
            pop %rax
            cmp %rbx, %rax
            jge __for_end_0000
        __for_body_0000:
            ...
        \end{minted}
        \end{block}
        \end{column}

        \pause{}

        \begin{column}{0.48\textwidth}
        \setbeamercolor{block title}{bg=green!60!black, fg=white}
        \setbeamercolor{block body}{bg=green!10, fg=black}
        \begin{block}{\t{fibonacci.jk} s\\ \t{-O signedness,annote}}
        \begin{minted}[fontsize=\tiny]{asm}
            ...
        __for_start_0000:
            movq -32(%rbp), %rax
            push %rax
            movq COUNT(%rip), %rax
            movq %rax, %rbx
            pop %rax
            cmp %rbx, %rax
            pushfq ## SIGNED TO UNSIGNED 0 START 
            pop %rax
            movq $2176, %rbx
            andq %rax, %rbx
            jz __clc_0000
            cmp $2176, %rbx
            jz __clc_0000
            orq $1, %rax
            jmp __sToU_0000
        __clc_0000:
            andq $-2, %rax
        __sToU_0000:
            push %rax
            popfq ## SIGNED TO UNSIGNED 0 END 
            jae __for_end_0000
        __for_body_0000:
            ...;
        \end{minted}
        \end{block}
        \end{column}

    \end{columns}

\end{frame}

\begin{frame}
    \frametitle{\textbf{Nové prvky jazyka}}

    \begin{itemize}
        \pause{}
        \item \t{foreach}
        \pause{}
        \item \textbf{\t{redo}}
        \pause{}
        \item \textbf{\t{restart}}
    \end{itemize}

    \pause{}

    \Image{7.5}{0.9}{0.6}{for_example.pdf}

\end{frame}

%\begin{frame}[fragile]{Side-by-Side Assembly Examples}
%
%    \begin{columns}[T,onlytextwidth]
%
%        \begin{column}{0.48\textwidth}
%        \begin{block}{Example 1: Move and Add}
%        \begin{minted}[fontsize=\scriptsize]{asm}
%        .intel_syntax noprefix
%        mov rax, 5
%        add rax, 10
%        mov rbx, rax
%        \end{minted}
%        \end{block}
%        \end{column}
%
%        \begin{column}{0.48\textwidth}
%        \setbeamercolor{block title}{bg=green!60!black, fg=white}
%        \setbeamercolor{block body}{bg=green!10, fg=black}
%        \begin{block}{Example 2: Loop Structure}
%        \begin{minted}[fontsize=\scriptsize]{asm}
%        .intel_syntax noprefix
%        mov rcx, 10
%        loop_start:
%            dec rcx
%            jnz loop_start
%        \end{minted}
%        \end{block}
%        \end{column}
%
%    \end{columns}
%
%\end{frame}

\begin{frame}
    \frametitle{Díky za pozornost!}

    \pause{}
    \centering{A jaké máte s obfuskací zkušenosti vy?}

\end{frame}

\end{document}
