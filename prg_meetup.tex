\documentclass[]{beamer}
    \usepackage[czech]{babel}
    \usepackage[utf8]{inputenc}

    \usetheme{Madrid}
    \usecolortheme{default}
    %\usefonttheme{wolverine}
    \setbeamertemplate{blocks}[rounded][shadow=true]

    \title[Základy obfuskace]{Základy obfuskace}
    \subtitle[]{aneb jak vývojáři efektivně zkazit den}
    \author[]{Bc.\ Martin Jabůrek}
    \institute[PRG PRG]{Student Programming Language Meetup}
    \date[]{28.\ listopadu\ 2025}

\begin{document}

\frame{\titlepage}

\begin{frame}
    \frametitle{Jak to začalo?}

    fotky ze screenhotů s Adamem co jsem našel (IAN (vyrval jsem v projektu 3 ten zdroják)
    to BP (překladač))
    
\end{frame}

\begin{frame}
    \frametitle{Jak to dopadlo?}

    screenshot stránky s BP\\
    screenshot co psala Verča u státnic

\end{frame}

\begin{frame}
    \frametitle{Ochrana proti zpětné analýze}

    proč?

\end{frame}

\begin{frame}
    \frametitle{A jak jsem to udělal já?}

    Jabukód

\end{frame}

\begin{frame}
    \frametitle{Metody}

    výpis a komentář

\end{frame}

\begin{frame}
    \frametitle{Obfuskace}

    teorie\\
    druhy\\
    metriky\\
    něco říct o revech (zmínit hantec)

\end{frame}

\begin{frame}
    \frametitle{Různé druhy obfuskace}

    na každou slide\\
    porovnání dvěma bloky asembleru co to vygeneruje\\
    nějaká moudrá slova k tomu a crowd participation

\end{frame}

\begin{frame}
    \frametitle{Titulek}

    \begin{block}{Blok}
        text
    \end{block}

    \pause{}

    \begin{itemize}
        \item položka
        \pause{}
        \item položka
    \end{itemize}
\end{frame}

\begin{frame}
    \frametitle{Crowd participation}

    zkušenosti lidí v praxi\\
    nenávistné poznámky od lidí co to museli řešit

\end{frame}

\end{document}
